\documentclass[12pt]{report}


\usepackage{fouriernc}%la fuente
%\usepackage[sc]{mathpazo} %antigua fuente

\usepackage[utf8]{inputenc}

\usepackage[a4paper,width=150mm,top=25mm,bottom=25mm]{geometry}



\usepackage{subfiles} %esto es para modularizar el overleaf
%para usar este paquete solamente hay que usar el comando
%\subfile{}



\usepackage{graphicx}

\graphicspath{{./Figuras/Teoria extremal de grafos/}{./Figuras/}{./Figuras/Numeros extremales en grafos bipartitos/}} %esto es para que encuentre las figuras hechas con pdf_tex en inkscape

\usepackage{framed}
\usepackage[dvipsnames]{xcolor} %agrega mas colores para xcolor.

%\usepackage[outdir=./]{epstopdf} %sin esto importar eps es imposible



\usepackage{xparse}
\usepackage{xstring}

\usepackage{stmaryrd} %para poner el comando \mapsfrom "<---|"

\usepackage{amssymb}

\usepackage{amsmath}

\usepackage{subfig}

\usepackage{mathrsfs} % para tener mas tipos de texto: \mathscr que es una letra mayuscula cursiva.

\usepackage{tikz-cd}

\usepackage{tkz-graph}%este paquete es para crear grafos con el ambiente \begin{tikzpicture}

\usepackage{caption}

\usepackage[shortlabels]{enumitem}

\usepackage{mathabx}
\let\widering\relax %esto es porque hay problemas con el comando \widering que se define en la fuenta fouriernc y en el paquete \usepackage{mathabx}

\usepackage[spanish,activeacute]{babel}

\usepackage{xparse}
\usepackage{xstring}

\usepackage{braket} %para definir \set , \Set y que los conjuntos se vean mas lindos

\usepackage{mathtools}

\usepackage[shortlabels]{enumitem}

\usepackage{hyperref}
\hypersetup{
    colorlinks,
    citecolor=red,
    filecolor=red,
    linkcolor=red,
    urlcolor=red
}

%%%%%%%%%%%%%%%%%%%%%%%%%%%%%%%%%%%%%%%%%%%%%
\usepackage{amsthm}

\theoremstyle{plain}
\newtheorem{theorem}{Teorema}[section]
\newtheorem{lemma}[theorem]{Lema}
\newtheorem{proposition}[theorem]{Proposición}
\newtheorem{proposition/definition}[theorem]{Proposición/Definición}
\newtheorem{corollary}[theorem]{Corolario}
\newtheorem{conjecture}[theorem]{Conjetura}
\newtheorem{afirmacion}[theorem]{Afirmación}
\newtheorem{recuerdo}[theorem]{Recuerdo}

\theoremstyle{definition}
\newtheorem{definition}[theorem]{Definición}
\newtheorem{hypothesis}[theorem]{Hipótesis}
\newtheorem{example}[theorem]{Ejemplo}
\newtheorem{obs}[theorem]{Observación}
\newtheorem{notation}[theorem]{Notación}
\newtheorem{remark}[theorem]{Comentario}


%por alguna razon el teorema $warning  est aen uso, asi que lo remuevo de maqnera trucha
\newtheorem{warn}[theorem]{\textbf{ADVERTENCIA}}
\renewenvironment{warning}{\begin{warn}}{\end{warn}}

%crear ejercicio
\newtheorem{exercise}[theorem]{Ejercicio}
%solución
\newenvironment{solution}{\begin{proof}[Solución]}{\end{proof}}





%como crear un nuevo ambiente de teorema o proposición que este sobreado con un recuadro de "color". primero hacemos

%\newenvironment{Theorem}{\colorlet{shadecolor}{color} \begin{shaded} \begin{theorem} }{ \end{theorem} \end{shaded} }

%Notar que primero hay que definir el color del sobreado con el comando
%"\colorlet{shadecolor}{color}" y luego hay que usar el environment "shaded". Adentro de este ponemos el environment que queremos, en nuestro caso queremos "pintar" el environment "\begin{theorem}".


%se puede cambiar la tonalidad de un color "yellow!80" es el color amarillo pero al 80%  y el 20% es mezclado con blanco, i.e. está aclarado. Pero "yellow!80!Black" es 80% amarillo y 20% negro, i.e. es obscurecido 20%.

\newenvironment{Definition}{\colorlet{shadecolor}{Apricot!12} \begin{shaded} \begin{definition} }{ \end{definition} \end{shaded} }

\newenvironment{Example}{\colorlet{shadecolor}{Goldenrod!16} \begin{shaded} \begin{example}}{ \end{example} \end{shaded}}

\newenvironment{Remark}{\colorlet{shadecolor}{Orchid!12} \begin{shaded} \begin{remark}}{ \end{remark} \end{shaded}}

\newenvironment{Warning}{\colorlet{shadecolor}{red!12} \begin{shaded} \begin{warning}}{ \end{warning} \end{shaded}}

\newenvironment{Conjecture}{\colorlet{shadecolor}{magenta!16} \begin{shaded} \begin{conjecture}}{ \end{conjecture} \end{shaded}}

\newenvironment{Theorem}{\colorlet{shadecolor}{OliveGreen!18} \begin{shaded} \begin{theorem}}{ \end{theorem} \end{shaded}}

\newenvironment{Lemma}{\colorlet{shadecolor}{LimeGreen!12} \begin{shaded} \begin{lemma}}{ \end{lemma} \end{shaded}}

\newenvironment{Proposition}{\colorlet{shadecolor}{Green!12} \begin{shaded} \begin{proposition}}{ \end{proposition}\end{shaded}}

\newenvironment{Corollary}{\colorlet{shadecolor}{TealBlue!16} \begin{shaded} \begin{corollary}}{ \end{corollary} \end{shaded}}

\newenvironment{Obs}{\colorlet{shadecolor}{Dandelion!22} \begin{shaded} \begin{obs}}{ \end{obs} \end{shaded}}

\newenvironment{Exercise}{\colorlet{shadecolor}{Lavender!12} \begin{shaded} \begin{exercise}}{ \end{exercise} \end{shaded}}

%%%%%COLORES%%%%%%%%%%%%
%Hay varios comandos del paquete Xcolor:
%\color{blue,green,red,yellow,orange,black,white,pink,purble,etc...} hace que todo el bloque de texto se transforme en este color, se puede encerrar entre {} el bloque de texto que uno quiere colorear
%\textcolor{color}{text} escribe el texto "text" en "color".
%\colorbox{color}{text} pinta un rectangulo de "color" detrás del "text".
%\shaded



%lista de colores base de xcolor, como son colores de la extension del paquetem, empiezan con la primera letra mayuscula: si usaramos solo el paquete {xcolor} entonces no sería necesario.

%red, Green (fluorecente), Blue (muy obscuro), Cyan, Magenta, Yellow, Black, Gray, lightgray, White, darkgray, lightgray, Brown, lime (este verde mas lindo manzana), olive (marron verdoso feo), Orange, pink, Purple, teal (verde marino), Violet

%marco los colores lindos: red, Cyan, Magenta, Yellow, Black, Gray, White,  lime, Orange, pink, teal, Violet

%Colores que incluye el paquete dvipsnames: Apricot (color beige), Brown, Goldenrod, JungleGreen, Salmon, Lavender, SpringGreen, Turquoise, Plum, Emerald, BurntOrange (naranja piola), ForestGreen (verde oscuro), BrickRed (rojo obscuro)


\newcommand{\red}[1]{\textcolor{BrickRed}{#1}}

			\newcommand{\comentario}[1]{\red{#1}}

\newcommand{\green}[1]{\textcolor{SpringGreen}{#1}}

\newcommand{\blue}[1]{\textcolor{Cyan}{#1}}

\newcommand{\darkblue}[1]{\textcolor{Cyan!70!Black}{#1}}

\newcommand{\yellow}[1]{\textcolor{yellow!80!Black}{#1}} %se puede cambiar la tonalidad de un color "yellow!80" es el color amarillo pero al 80%  y el 20% es mezclado con blanco, i.e. está aclarado. Pero "yellow!80!Black" es 80% amarillo y 20% negro, i.e. es obscurecido 20%.

\newcommand{\black}[1]{\textcolor{Black}{#1}}

\newcommand{\gray}[1]{\textcolor{Gray}{#1}}

\newcommand{\purple}[1]{\textcolor{Purple}{#1}}

\newcommand{\beige}[1]{\textcolor{Apricot}{#1}}

\newcommand{\darkgreen}[1]{\textcolor{ForestGreen}{#1}}

\newcommand{\pink}[1]{\textcolor{Lavender}{#1}}

\newcommand{\salmon}[1]{\textcolor{Salmon}{#1}}

\newcommand{\brown}[1]{\textcolor{RawSienna!50!Black}{#1}}

\newcommand{\white}[1]{\textcolor{White}{#1}}

\newcommand{\orange}[1]{\textcolor{BurntOrange}{#1}}













%%%%%%%%%%%%%%%%%%%%%%%%%%%%%%%%%%%%%%%%%%%%%




%grupos de matrices
%SL
\newcommand{\SL}[2]{\operatorname{SL}_{#1} ( #2)}
%GL
\newcommand{\GL}[2]{\operatorname{GL}_{#1} ( #2)}

%matriz identidad
\newcommand{\Id}{\operatorname{Id}}



%enteros Z
\newcommand{\integers}{\mathbb{Z}}
%racionales
\newcommand{\rationals}{\mathbb{Q}}
%naturales
\newcommand{\naturals}{\mathbb{N}}
%reales R
\newcommand{\reals}{\mathbb{R}}
%imaginarios
\newcommand{\complex}{\mathbb{C}}
%p-adicos
\newcommand{\padics}{\mathbb{Q}_p}
%enteros p-adicos
\newcommand{\padicintegers}{\mathbb{Z}_p}

%cuerpos finitos
%Fp
\newcommand{\Fp}{\mathbb{F}_p}
%Fq
\newcommand{\Fq}{\mathbb{F}_q}



%valor absoluto p-adico
\newcommand{\abs}[1]{\left \vert #1 \right \vert}
%valor absoluto p-adico
\newcommand{\Abs}[1]{\left \vert \left \vert #1 \right \vert \right \vert}
%valuacion p-adica
\newcommand{\val}[1]{\operatorname{val} (#1)}

%Hom
\newcommand{\Hom}{\operatorname{Hom}}

%imagen y núcleo
\newcommand{\Imagen}{\operatorname{Im}}
\newcommand{\Ker}{\operatorname{Ker}}

%coker
\newcommand{\Coker}{\operatorname{Coker}}

%limite inverso
\newcommand{\liminv}{\varprojlim}


%un poco de typeset para categorias
\newcommand{\catname}[1]{{\operatorfont\textbf{#1}}}


\renewcommand{\hat}[1]{\widehat{#1}}
\renewcommand{\bar}[1]{\overline{#1}}

%declaro un comando nuevo para escribir restricción de funciones
\newcommand\rest[2]{{% we make the whole thing an ordinary symbol
  \left.\kern-\nulldelimiterspace % automatically resize the bar with \right
  #1 % the function
  \vphantom{\big|} % pretend it's a little taller at normal size
  \right|_{#2} % this is the delimiter
  }}


%%%%   COMANDO ALGEBRA CONMUTATIVA   %%%%

%altura de un ideal:
\newcommand{\height}{\textsc{height}}

%Clausura topológica
\newcommand{\closure}[1]{\overline{#1}}

%longitud de un A-modulo. Notacion: \length_A M
\newcommand{\length}{\operatorname{length}}

%Anulador de un $A$-módulo.
\newcommand{\Ann}[1]{\operatorname{Ann} (#1)}

%Cuerpo de fracciones. Notacion $\FracField A$.
\newcommand{\FracField}[1]{\operatorname{Fr} (#1)}


%%%%%%%%%%%%%%%%%%%%%%%%%%%%%%%%%%%%






%%%%   COMANDO TEORÍA DE NÚMEROS  %%%%

%Discriminante
\newcommand{\discriminant}[1]{\mathfrak{d} (#1 )}

%%%%Ideales primos%%%
%escribe una letra en notación mathfrak, para denotar a un ideal o elemento primo.

\newcommand{\primo}[1]{\mathfrak{#1}}
\newcommand{\Primo}[1]{\mathfrak{\MakeUppercase{#1}}}

%anillo de enteros O_K
\renewcommand{\O}{\mathcal{O}}
%anillo de enteros con subindice de cuerpo (input, por ejemplo $K$).
\newcommand{\integralring}[1]{O_{#1}}

%caracteristica de un cuerpo Char k
\newcommand{\Char}[1]{\operatorname{Char} #1}

%traza. Notación \trace = Tr
\newcommand{\trace}{\operatorname{Tr}}

%Traza de extensiones. Notación \Tr L K \alpha = \operatorname{Tr}_{L/K} (\alpha)
\newcommand{\Tr}[1]{\operatorname{Tr}_{L/K} (#1)} %la extension es L/K por default
\newcommand{\tr}[3]{\operatorname{Tr}_{#1/#2} (#3)}

%Norma de extensiones. Notación \Norm L K \alpha = \operatorname{N}_{L/K} (\alpha)
\newcommand{\Norm}[1]{\operatorname{N}_{L/K} (#1)}%la extension es L/K por default
\newcommand{\norm}[3]{\operatorname{N}_{#1/#2} (#3)}


%discriminante de una forma bilineal simetrica. notacion \disc{B} = \operatorname{disc} ( B)
\newcommand{\disc}[1]{\operatorname{disc} (#1)}

%%%%%%%%%%%%%%%%%%%%%%%%%%%%%%%%%%%%




%%%%%%%%%%%%%COMANDO GRAFOS%%%%%%%%%%%%%

%\ceil funcion techo
\newcommand{\ceil}[1]{\left\lceil #1  \right\rceil}

%\floor funcion piso
\newcommand{\floor}[1]{\left\lfloor #1  \right\rfloor}

%diámetro de un grafo
\newcommand{\diam}[1]{\operatorname{diam} (#1)}

%radio de un grafo
\newcommand{\rad}[1]{\operatorname{rad}(#1)}

%Kappa:
\newcommand{\Kappa}{\mathcal{K}}

%Defecto:
\newcommand{\defecto}[1]{\mathrm{df}(#1)}

%Conjunto de últimos vértices de una familia \mathcal P de caminos dirigidos: \ter{\mathcal P}
\newcommand{\ter}[1]{\operatorname{ter} (#1)}

%numero de coloreo de un grafo G:
\newcommand{\col}[1]{\operatorname{col} (#1)}

%número de lista coloreo de un grafo G:
\newcommand{\ch}[1]{\operatorname{ch} (#1)}



%Número extremal
\newcommand{\ex}[2]{\operatorname{ex} (#1, #2)}









%%%%%%%%%%%%%%%%%%%%%%%%%%%%%%%%%%%%



%%%%%%%%%%%%%%%%%%%%%%%%%%%%%%
\newcounter{numeroSeccion}[section]%ponemos un contador que empieza en 0 y que cuenta el número de seccion

\newcounter{numeroCapitulo}[chapter]

\newcounter{numeroDibujo}[numeroSeccion]


%%%%%%%%%%%%%%%%%%%%%%%%%%%%%%
%Cada dibujo se puede automatizar:
%1) necesitamos el archivo "Dibujo n.png" en la carpeta "Clase m", donde $n$ es el número del dibujo y $m$ es el número de la clase.



\renewcommand\thefigure{\thesection.\arabic{figure}}

%el comando Dibujo tiene dos inputs \Dibujo{input 1}{input 2}, el primer input es [OPCIONAL] y representa el ancho del dibujo, y el segundo es el caption de la figura.
\NewDocumentCommand{\Dibujo}{O{1} m m}{
\stepcounter{numeroDibujo}
\begin{center}\label{Figura:Capitulo \thenumeroCapitulo - Seccion \thenumeroSeccion Dibujo \thenumeroDibujo}
\includegraphics[width=#1\columnwidth]{#3}
\captionof{figure}{#2}
\end{center}
}


%el comando Inkscape tiene dos inputs \Inkscape{input 1}{input 2}, el primer input es [OPCIONAL] y representa el ancho del dibujo, y el segundo es el caption de la figura.
\NewDocumentCommand{\Inkscape}{O{1} m m}{
\stepcounter{numeroDibujo}
\begin{center}\label{Figura:Capitulo \thenumeroCapitulo - Seccion \thenumeroSeccion Dibujo \thenumeroDibujo}
\def\svgwidth{#1\textwidth}
\input{#3}
\captionof{figure}{#2}
\end{center}
}


%%%%%%%%%%%%%%%%%%%%%%%%%%%%%%%

\title{Apuntes - Tópicos en matemática discreta}
\author{Enzo Giannotta}






\begin{document}

\maketitle

%--------------------------------- ACA VA LA TABLA DE CONTENIDOS

\tableofcontents

%---------------------------------






\chapter{Teoría extremal de grafos}

En este curso trabajaremos con grafos simples, usualmente denotados: $G=(V,E)$.

\section{Teoría extremal de grafos}

¿Cuál es la máxima cantidad de aristas que puede tener un grafo de $n$ vértices sin que aparezca una cierta estructura?

¿Cómo lucen estos grafos maximales?

\begin{example}
\begin{enumerate}
\item Cuando la estructura es un ciclo, la cantidad de aristas es $n-1$ y los grafos maximales son los árboles.
\item Cuando la estructura es un ciclo impar. ¿Cómo lucen los grafos sin ciclos impares y que tienen una cantidad máxima de aristas? Son los completos balanceados $K_{\ceil {\frac n 2},\floor{\frac n 2}}$. En efecto, para que un grafo bipartito con $n$ vértices tenga una cantidad máxima de aristas, tiene dos partes $\abs X, \abs Y$ con $\abs X + \abs Y = n$ y si maximiza la cantidad de aristas es un grafo $K_{\abs X, \abs Y}$. Es decir, tiene $\abs X \cdot \abs Y$ aristas y si maximizamos, hay que maximizar la función $f(y) = (n-y)y$ con $1 \leq y \leq n-1$ e $y$ entero; esto sucede sii $y = \lfloor {\frac n 2} \rfloor$ o $y = \lceil \frac n 2 \rceil$.
\end{enumerate}
\end{example}

\begin{definition}
Sean $G$ y $H$ dos grafos. Decimos que $G$ es \textbf{H}-libre (o \textbf{libre de $H$}) si $H \not \subset G$. El \textbf{número extremal} de $H$ es la cantidad
\[
    \ex n H = \max \{e (G) | G \text{ es un grafo de $n$ vértices $H$-libre}\},
\]
donde $e(G)$ siempre denotará el número de aristas de $G$.

Si $G$ es $H$-libre y $\Abs G = \ex n H$, decimos que $G$ es \textbf{extremal} respecto de $n$ y $H$.
\end{definition}


\begin{theorem}[Mantel, 1907]
Sea $n \in \naturals$, $G$ un grafo $K_3$-libre con $n$ vértices. Entonces, $e(G) \leq \ceil{\frac n 2} \floor {\frac n 2}$. Además, $e(G) = \ceil{\frac n 2} \floor {\frac n 2} \Leftrightarrow G = K_{\ceil{\frac n 2}, \floor {\frac n 2}}$\footnote{Cuando $n = 1,2$ tenemos que $G$ es el completo $K_n$}.
\end{theorem}
\begin{proof}
Por inducción en $n$. Los casos $n = 1, n=2$ son un vértice, un $1$-camino respectivamente. Luego vale para $n=1,2$. Ahora, supongamos que $n \geq 3$. Sea $G$ un grafo $K_3$-libre con $n$ vértices, y $uv \in E(G)$ (si $G$ no tuviera aristas, podríamos agregar una arista y seguiría siendo $K_3$-libre); consideremos $G' = G \setminus \{u,v\}$. Tenemos que $G'$ también es $K_3$-libre y tiene $n-2$ vértices. Por inducción, $G'$ satisface
\[
    e(G') \leq \ceil{\frac {n-2} 2} \floor {\frac {n-2} 2}.
\]
Más aún, como $G$ es $K_3$-libre, no existen vértices $w \in G'$ tal que sea adyacente a $u$ y $v$ al mismo tiempo. Luego existen a lo más $n-2$ aristas en $E(G) \setminus E(G')$ sin contar la arista $uv$. Es decir,
\[
    e(G) \leq e(G') + n-1 \leq \ceil{\frac n 2} \floor {\frac n 2}.
\]
\Inkscape{Ilustración}{"./Figuras/Teoria extremal de grafos/Dibujo 1.pdf_tex"}


Para la segunda parte, $e(G) =\ceil{\frac n 2} \floor {\frac n 2} \Leftrightarrow G = K_{\ceil {\frac{n}{2}} , \floor {\frac{n}{2}}}$. Es claro que si $G = K_{\ceil{\frac n 2}, \floor {\frac n 2}}$ luego $e (G) = \ceil{\frac n 2}, \floor {\frac n 2}$. Veamos la recíproca. Sea $G$ con $n$ vértices y cantidad máxima de aristas tal que es $K_3$-libre. Los casos $n=1,2$ son triviales, luego podemos suponer que $\abs G \geq 3$. Como $G$ es $K_3$-libre, existen una aristas $uv \in E(G)$ por maximalidad. Por inducción, $G':= G \setminus \{u,v\}$ es un $K_{\ceil{\frac {n-2} 2}, \floor {\frac {n-2} 2}}$, digamos con partición $X',Y' \subset V(G')$ de sus vértices. Como $G$ es $K_3$-libre, ni $u$ ni $v$ pueden tener vecinos en $G'$ que estén en ambas particiones $X',Y'$, además, no puede haber una partición que no tenga a $u$ y $v$ como vecinos en $G$ pues podríamos agregar aristas entre vértices de esa particiones: contradiciendo maximalidad. Sin pérdida de generalidad, los vecinos de $u$ en $G'$ están en $X$ y los de $v$ en $Y$. Más aún, por maximalidad, todos los vértices de $X$ son vecinos con $u$ y todos los de $Y$ con $v$. Así, $G$ es un $X,Y$ bigrafo tomando $X := X' \cup \{v\}$ e $Y := Y' \cup \{u\}$. Notar que esto prueba que $G$ es un $K_{\ceil{\frac n 2}, \floor {\frac n 2}}$.


\end{proof}


\begin{definition}
El \textbf{grafo de Turán} $T_k (n)$ es el grafo $k$-partito completo con la mayor cantidad de aristas, es decir, los cardinales de las particiones difieren a lo más en $1$ entre sí (por maximalidad). Notamos
\[
    t_k(n) := e(T_k (n)).
\]
\end{definition}

\begin{obs}
Podemos calcular $t_k(n)$. Sea $\alpha \in \naturals$ el cardinal más grande de una partición de $T_k (n)$. Entonces las demás particiones tienen cardinal $\alpha$ o $\alpha -1$. Sea $r$ la cantidad de particiones con cardinal $\alpha -1$ y $k-r$ de cardinal $\alpha$. Tenemos que sumando los cardinales de todas las particiones:
\[
    \alpha k - r = n.
\]
Como $0 \leq r < k$, $r$ es el resto de la división de $n$ por $k$ y $\alpha$ es el cociente. Despejando obtenemos que $\alpha = \frac{n+r}{k}$ es decir, $\alpha = \lceil \frac{n}{k} \rceil$. En particular $\alpha -1 = \lfloor \frac{n}{k} \rfloor$. Juntado todo, tenemos que la cantidad total de aristas es:
\[
    \alpha^2 \binom{k-r}2 + \alpha (\alpha -1) (k-r)r + (\alpha-1)^2 \binom{r}2,
\]
i.e.,
\[
    \boxed{t_k(n) = \lceil \frac{n}{k} \rceil^2 \binom{k-r}2 + \lceil \frac{n}{k} \rceil \lfloor \frac{n}{k} \rfloor (k-r) r + \lceil \frac{n}{k} \rceil^2 \binom r 2.}
\]
\end{obs}


\begin{theorem}[Turán, 1941]\label{th:teorema de Turan todo grafo K_k+1 libre en n vertices tiene cantidad de aristas e < t_k (n)}
Sean $n,k \in \naturals$, $G$ un grafo $K_{k+1}$-libre con $n$ vértice. Entonces
\[
    e(G) \leq t_k (n).
\]
Además, $e(G) = t_k (n) \Leftrightarrow G = T_k (n)$\footnote{Cuando $n = 1,2, \ldots, k-1$ tenemos que $G$ es el completo $K_n$}.
\end{theorem}

\begin{proof}
Hagamos inducción en $n$. Para $n \leq k$ es trivial. Sea ahora $G$ con $n \geq k+1$ que a su vez es $K_{k+1}$-libre y arista maximal. Esto implica que agregar cualquier arista hace aparecer un $K_{k+1}$ como subgrafo. Entonces $G$ contiene un $K_{k}$. Sea $A$ el conjunto de vértices de un subgrafo $K_{k}$ en $G$. Consideremos luego $G' = G\setminus A$. El grafo $G'$ es $K_{k+1}$-libre y tiene $n-k$ vértices. Cada $x \in V(G')$ tiene a lo más $k-1$ vecinos en $A$ dentro del grafo $G$, pues $G$ es $K_{k+1}$-libre. Luego por hipótesis inductiva:
\[
    e(G') \leq t_k(n-k).
\]
Si juntamos esto con la hipotesis inductiva, tenemos que
\[
    e(G) \leq e(G') + (n-k) (k-1) + \binom {k} 2    \leq t_k (n-k) + (n-k) \cdot (k-1) + \binom {k} 2 = t_k (n),
\]
donde el segundo término es la cantidad de aristas entre $A$ y $V(G')$.

Veamos ahora la segunda afirmación. Por definición, $G = T_k (n)$ tiene $t_k (n)$ aristas. Recíprocamente, supongamos que $G$ con $n$ vértices y cantidad máxima de aristas $e(G)$ tal que es $K_{k+1}$-libre. Los casos $n \leq k$ son triviales, luego supongamos que $n \geq k+1$. Por maximalidad, $G$ contiene un $K_{k}$ como subgrafo; llamemos $A$ a su conjunto de vértices en $G$ y consideremos $G' := G \setminus A$. Notar que
\[
    e(G') \geq e(G)- \left ( (n-k) (k-1) + \binom {k} 2 \right ) = t_k (n) -(n-k) (k-1) - \binom{k} 2 = t_k(n-k),
\]
pues cada vértice de $G'$ tiene a lo más $k-1$ vecinos en $A$. Como $G'$ es $K_{k+1}$-libre, en realidad vale la igualdad: $e(G') = t_k (n-k)$, por la primera parte que ya demostramos. Llamemos $X_1, X_2, \ldots, X_k$ a las particiones de $G'$. Como vale la igualdad arriba, tenemos que cada vértice de $G'$ tiene exactamente $k-1$ vecinos en $A$. Para cada $x' \in G'$ llamemos $\alpha(x')$ al único vértice de $A$ que no es adyacente a $x'$ en $G$. Más formalmente, $\alpha : V(G') \rightarrow A$ es una función; afirmamos que:
\begin{enumerate}[(i)]
\item $\alpha$ es sobreyectiva.
\item Si $x_i ' \in X_i$ y $X_j ' \in X_j$ para $i \neq j$, entonces $\alpha (x_i') \neq \alpha (x_j')$.
\end{enumerate}
Antes de probar la afirmación, notemos que esta prueba que
$\rest{\alpha}{X_i}$ es constante para cada $i = 1, \ldots, k$ (y por lo tanto tiene sentido el abuso de notación $\alpha (X_i)$ para denotar al único vértice de $A$ que no es adyacente a ningún vértice $x' \in X_i$). Veamos entonces la afirmación:
\begin{enumerate}[(i)]
\item Supongamos que $\alpha$ no es sobreyectiva: existe un $a_0 \in A$ tal que para todo $i = 1, \ldots, k$ existe $x_i ' \in X_i$ adyacente a $a_0$ en $G$. Pero esto implica entonces que los vértices $x_1' , \ldots, x_k ', a_0$ forman un $K_{k+1}$ en $G$, absurdo.
\item En efecto, si $\alpha (x_i ') = a_0 = \alpha (x_j ')$, entonces $x_i, x_j$ y los vértices de $A \setminus \{a_0\}$ juntos forman un $K_{k+1}$ en $G$, absurdo.
\end{enumerate}

Así, podemos extender la partición de $G'$ a todo $G$: definimos $\tilde X_i := X_i \cup \{\alpha (X_i)\}$. Es claro que de esta manera $G$ es un grafo $k$-partito completo. Como $G$ es maximal en su cantidad de aristas, entonces $G = T_k (n)$.
\end{proof}




\begin{theorem}[Erdös - segunda demostración del teorema]
Sean $n, k  \in \naturals    $ y $G$ un grafo $K_{k+1}$-libre con $n$ vértices. Entonces existe un grafo $H$ que es $k$-partito con $V(H) = V(G)$ tal que:
\[
    d_H (v) \geq d_G (v) , \quad \forall v \in V(G).
\]
\end{theorem}
\begin{proof}[Erdös]
Haremos inducción en $k$. Para $k = 1$ no hay que hacer nada. Sea ahora $k \geq 2$. Sea $v \in V(G)$ con $d_G (v) = \Delta (G)$. La vecindad de $v$, $G' := G[N_G (v)]$ debe ser $K_{k}$-libre. Sea $A := G \setminus N_G (v)$. Notar que
\[
    d_{G} (u) \leq d_{G'} (u) + \abs A.
\]
Por hipótesis inductiva existe un grafo $H'$ que es $(k-1)$-partito con $V(H') = V(G')$ y
\[
    d_{H'} (u) \geq d_{G'} (u), \quad \forall u \in V(G').
\]
Sea $H$ el grafo obtenido a paratir de $H'$ añadiendo los vértices de $A$ y conectando todas las aristas entre $A$ y $V(H')$. Observar que $H$ es $k+1$-partito y como $v$ tiene grado máximo en $G$, tenemos que para cada $u \in A$:
\[
    d_G (u) \leq d_G (v) = \abs{V(H')} = d_H (u)
\]
y para $u \in V(H')$ sabemos que:
\[
    d_G (u) \leq d_{G'} (u) + \abs A    \underset{H.I.}{\leq} d_{H'} (u) + \abs A = d_H (u).
\]
\end{proof}

\begin{exercise}
A partir de la demostración deducir que el grafo $K_{k+1}$-extremal es $T_k (n)$ y es único.
\end{exercise}

\begin{obs}
Sea $H$ un grafo con $\chi (H) \geq 3$, es decir no bipartito, entonces
\[
    \ex n H = \Theta (n^2).
\]
\end{obs}
\begin{proof}
En primer lugar, si $G$ es un grafo que no contiene a $H$ luego no puede ser bipartito; en particular si $G = K_{\ceil { \frac n 2} , \floor {\frac n 2}}$ entonces tiene $n$ vértices y $e(G) = \ceil { \frac n 2} \floor {\frac n 2}$. Consecuentemente
\[
    (n-1)^2/4 \leq  \ceil { \frac n 2} \floor {\frac n 2} \leq \ex n H.
\]

Por otro lado, la cantidad de aristas maxima de $G$ es $\binom n 2$ (en general para cualquier grafo con $n$ vértices) y por lo tanto $\ex n H = \Theta (n^2)$.
\end{proof}




\section{Números extremales en grafos bipartitos}


\begin{recuerdo}[Desigualdad de Jensen]
Vamos a usar la desigualdad de Jensen: si $\varphi$ es una función convexa entonces:
\[
    \boxed{\varphi(\mathbb E (X)) \leq \mathbb E (\varphi (X)).}
\]
\end{recuerdo}


\begin{exercise}
Probar las siguientes dos desigualdades elementales para el binomio de Newton:
\[
    \left(\frac n k \right)^k \overset{\text{Cota 1}}{\leq} \binom n k \overset{\text{Cota 1}}{\leq} \left(\frac{n \cdot e}{k}\right)^k.
\]
\end{exercise}
\begin{solution}
\begin{enumerate}
\item[Cota 1:] Recordar que el binomio de Newton tiene la siguiente identidad recursiva: $\binom n k = $
\item[Cota 2:]
\end{enumerate}
\end{solution}

n/k < n-1 /k-1 si y solo si n k-1 < n -1 k sii -n < -k sii k < n


\begin{theorem}[Erdös, 1938]
Para todo $n \in \naturals$
\[
    \ex n {C_4} \leq n^{\frac 3 2}.
\]
\end{theorem}

\begin{definition}
Una \textbf{cereza} es un $2$-camino $x_0 x_1 x_2$. Llamaremos a $x_1$ el \textbf{centro} y a $x_0,x_2$ las \textbf{hojas}.
\end{definition}

\Inkscape{Dibujo de cereza.}{"./Figuras/Numeros extremales en grafos bipartitos/Dibujo 1.pdf_tex"}

\begin{proof}
Sea $G$ un grafo $C_4$-libre con $n$ vértices. Contaremos cereza en $G$ para acotar el número de aristas $e(G)$.

Para cada vértice $v \in V(G)$ hay exactamente
\[
    \binom{d(v)}{2} \text{ cerezas con centro en $v$}.
\]
Por lo tanto, en $G$ hay
\[
    \sum_{v \in V(G)} \binom{d(v)}{2} \text{ cerezas en $G$}.
\]
Por la desigualdad de Jensen la sumatoria se minimiza cuando todos los grados son iguales:
\begin{align*}
    \sum_{v \in V(G)} \binom{d(v)}{2} &\geq n \cdot \binom{2 e(g) / n}{2} \\
                                        &\overset{Cota 1}{\geq} n \cdot \left ( \frac{e(G)}{n} \right )^2 = \frac{e(G)^2}{n}.
\end{align*}
Por otro lado, dado un par $\{u,v\}$ de hojas de cerezas distintas, entonces tendríamos un subgrafo $C_4$ en $G$, absurdo; por lo tanto hay a lo más
\[
    \binom{n} 2 \text{ cerezas en $G$}.
\]

Juntando todo:
\[
    \frac{e(G)^2}{n} \leq \binom n 2 = \frac{n(n-1)}{2},
\]
consecuentemente $e(G)^2 \leq n^3$, i.e., $e(G) \leq n^{\frac{3}{2}}$.
\end{proof}




\begin{theorem}[Kövani, Sós, Turán]
Sean $s, t \in \naturals$, $s \leq t$. Entonces existe una constante $c = c(s,t) > 0$ tal que
\[
    \ex n {K_{s,t}} \leq c \cdot n^{2 - \frac 1 s}, \quad \forall n \in \naturals.
\]
\end{theorem}

\begin{definition}
Una \textbf{$s$-cereza} es un $K_{1,s}$. Similarmente tenemos la noción de \textbf{centro} y \textbf{hojas} (las cuales son $s$).
\Inkscape{Dibujo de $s$-cereza.}{"./Figuras/Numeros extremales en grafos bipartitos/Dibujo 2.pdf_tex"}
\end{definition}

\begin{proof}
Sea $G$ un grafo $K_{s,t}$-libre en $n$ vértices. Para cada $v \in V(G)$ hay $\binom{d(v)}{s}$ $s$-cerezas. Por lo tanto en $G$ hay
\[
    \sum_{v \in V(G)} \binom{d(v)}{s} \text{ $s$-cerezas},
\]
con lo cual
\[
    \sum_{v\in V(G)} \binom{d(v)}{s} \overset{Jensen}\geq n \binom{2 e(G)/n}{s} \overset{Cota 1}\geq n \left( \frac{2 e (G)}{s n}\right)^2.
\]
Procediendo de manera análoga a la demostración del teorema anterior, tenemos que un conjunto de $s$ vértices del grafo puede ser conjunto de hojas de a lo más $(t-1)$ cerezas, pues de lo contrario habría una copia de $K_{s,t}$. Por lo tanto, hay en total a lo más
\[
    (t-1) \cdot \binom{n}s \text{ $s$-cerezas}.
\]

Juntando todo:
\[
    n ( \frac{2 e(G)}{s n})^s \leq (t-1) \cdot \binom n s \overset{Cota 2}{\leq} (t-1) \cdot (\frac{n e}{s})^s,
\]
luego
\[
    \frac{2 e(G)}{s n} \leq \frac{(t-1)^{\frac 1 s}}{n^{\frac 1 s}} \cdot \frac{n e}{s},
\]
equivalentemente,
\[
    e(G) \leq {\frac{(t-1)^{\frac 1 s} s e}{2 s}} \cdot n^{2-\frac{1}{s}} = c(s,t ) \cdot n ^{2 - \frac 1 s}.
\]
\end{proof}

\begin{exercise}
Demostrar que
\[
    \ex n H = o (n^2) \quad \Leftrightarrow \quad H \text{ es bipartito}.
\]
\end{exercise}



\section{Números extremales para árboles}



\begin{theorem}
Sean $n,k \in \naturals$ y $T$ un árbol con $k+1$ vértices. Entonces,
\[
    \ex n T \leq (k-1) \cdot n.
\]
\end{theorem}

\begin{lemma}
Sean $k \in \naturals$ y $T$ un árbol con $k+1$ vértices. Entonces si $G$ es un grafo con $\delta (G) \geq k$, luego contiene a $T$ como subfrafo.
\end{lemma}
\begin{proof}
Haremos inducción en $k$. Para $k = 1$ es claro, pues existe un vértice con al menos un vecino. En general, supongamos que $k \geq 2$. Sea $h$ una hoja de $T$ y consideremos el árbol $T' = T \setminus \{h\}$. Por hipótesis inductiva, $T' \subset G$. Sea $p$ el único vecino de $h$ en $T$, i.e. $p \in T'$. Como $T$ tiene $k+1$ vértices, $p$ tiene a lo más $k-1$ vecinos en $T'$, luego $p$ tiene un vecino en $G$ que no está en $T'$ pues $\delta_G (p) \geq k$. Entonces podemos incrustar $T$ en $G$ considerando $h$ como este vértice.
\end{proof}

\begin{lemma}
Todo grafo $G$ contiene un subgrafo $H$ con $\delta (H) \geq \frac{e(G)}{n}$, donde $n = \abs G$.
\end{lemma}
\begin{proof}
Ver Diestel.
\end{proof}

\begin{proof}[Demostración del teorema]
Sea $G$ un grafo con $\geq (k-1) \cdot n + 1$ aristas que no contiene a $T$. Por el segundo lema, $G$ contiene $H$ con
\[
    \delta (H) \geq \frac{e(G)}{n} > \frac{(k-1) n}{n},
\]
y por el primer lema ganamos.
\end{proof}

\begin{conjecture}[Erdös, Sós, 1963]
Se conjetura que en el teorema anterior se tiene una mejor cota:
\[
    \ex n T \leq \frac 1 2 (k-1) n.
\]

Notar que de ser verdadera la conjetura, entonces esta cota es tight cuando $n$ es un múltiplo de $k$: Sea $G$ el grafo obtenido al unir $\frac n k$ copias de $K_k$, así $e(G) = \frac n k \binom k 2 = \frac n 2 (k-1)$.

Esta conjetura es verdadera en el caso \underline{$T$ un camino}:
\end{conjecture}

\begin{theorem}[Erdös \& Gallai, 1959]
Sean $n, k \in \naturals$. Entonces,
\[
    \ex n {P_k} \leq \frac{(k-1) \cdot n}{2}
\]
\end{theorem}
\begin{exercise}
A partir de la demostración de este teorema, obtenga que los grafos extremales son únicos.
\end{exercise}

\begin{lemma}\label{lema:grafo G conexo con n vertices tiene un camino de largo al menos 2 delta o n-1}
Todo grafo conexo $G$ con $n$ vértices contiene un camino de largo
\[
    k := \min \{2 \delta (G), n-1\}.
\]
\end{lemma}
\begin{proof}
Tomemos $P := v_0, \ldots, ,v_l$ camino de largo máximo. Sabemos que $N_G (v_0), N_G (v_l) \subset V(P)$ por maximalidad de $P$. Si $V(P) = V(G)$ ganamos. Así que supongamos que no; supongamos también que $l < k\leq 2\delta (G)$. Demostraremos que existe un ciclo de longitud $l$ contenido en $G[V(P)]$, así llegaremos a una contradicción pues al existir un vértice $x$ fuera de $G[V(P)]$ en $G$, podríamos extender el ciclo a un camino de longitud al menos $k+1$ en $G$ conectándolo con $x$.

\Inkscape{Notar que en este caso $v_0 P v_{i-1} v_l P v_i v_0$ es un ciclo de longitud $\abs P$ en $G[V(P)]$.}{"./Figuras/Numeros extremales para arboles/figura.pdf_tex"}

En efecto, supongamos que no existe tal ciclo, luego para cada $i \in \{1, \ldots, l-1\}$ se tiene que $v_{i-1}v_l \not \in E(G)$ o $v_0 v_i \not \in E(G)$. Entonces
\[
    2 \delta(G) \leq d_G (v_0) + d_G (v_l) \leq l < 2 \delta (G),
\]
absurdo.
\end{proof}


\begin{proof}[Demostración del teorema]
Haremos inducción en $n$. Afirmamos que $G$ es $P_k$-libre en $n$ vérties, entonces
\[
    e(G) \leq \frac{(k-1)\cdot n}{2}.
\]
El caso base es $n \leq k$, luego $e(G) \leq \binom n 2 = \frac{n (n-1)}{2} \leq \frac{n (k-1)}{2}$. Luego supongamos que $n \geq k+1$. Si $G$ no es conexo: sean $G_1, \ldots, G_r$ las componentes conexas, por hipótesis
\[
    e(G_i) \leq \frac{\abs{G_i} (k-1)}{2},
\]
entonces
\[
    e(G) = \sum_{i = 1}^r e(G_i) \leq \frac{k-1}{2} \sum_{i =1}^r \abs {G_i} = \frac{n (k-1)}{2}.
\]

Ahora, supongamos que $G$ es conexo. Si $n-1 \leq 2 \delta (G)$, entonces por el Lema \ref{lema:grafo G conexo con n vertices tiene un camino de largo al menos 2 delta o n-1}, $G$ contiene un camino de largo $n-1 \geq k$, absurdo. Con lo cual, podemos asumir que $2 \delta (G) \leq n-1$, y por el Lema, $G$ contiene un camino de largo $2 \delta (G)$ que debe cumplir
\[
    2\delta (G) < k \quad \Leftrightarrow \quad \delta (G) \leq \frac{k-1}{2}.
\]
Sea $v$ un vértice de grado $\leq \frac{k-1}{2}$, consideremos $G' := G \setminus \{v\}$. Por hipótesis inductiva
\[
    e(G') \leq \frac{(n-1)(k-1)}{2},
\]
con lo cual,
\[
    e(G) \leq e(G') + \frac{k-1}{2} \leq \frac{(n-1)(k-1)}{2} + \frac{k-1}{2} = \frac{n(k-1)}{2}.
\]
\end{proof}



\section{Estabilidad y supersaturación}

\begin{theorem}[Füredi, 2015]\label{th:teorema de furedi si G esta t lejos de ser bipartito entonces tiene triangulos}
Sean $n,t \in \naturals$, y $G$ con $n$ vértices. Si $G$ está \textbf{$t$-lejos} de ser bipartito\footnote{Esto significa que si $H$ es un subgrafo bipartito de $G$, entonces $e(H) \leq e(G)-t$.}, entonces hay al menos
\[
    \frac{n}{6} \left(e(G)-\frac{n^2}{4}+t\right)
\]
triángulos en $G$.
\end{theorem}
\begin{proof}
Para cada $u \in V(G)$, definimos
\[
    B_u := N_G (u) \quad \text{y} \quad A_u := V(G) \setminus B_u.
\]
Luego la cantidad de tríangulos de $G$ es:
\[
    k_3 (G) = \frac 1 3 \sum_{u \in V(G)} e(B_u).
\]

Para cada $u \in V(G)$, si borro las aristas de $G[B_u]$ y las de $G[A_u]$, obtengo un subgrafo bipartito de $G$: el $(A_u,B_u)$-bigrafo; luego tuvimos que haber quitado al menos $t$ aristas porque $G$ está $t$-lejos de ser bipartito, es decir:
\[
    e(B_u) + e(A_u) \geq t.
\]

Además, para cada $u \in V(G)$
\[
    \sum_{v \in A_u} d_G (v) = e(B_u, A_u) + 2e (A_u).
\]
Como
\[
    e(G) = e(A_u) + e(A_u, B_u) + e(B_u),
\]
se sigue que $e (A_u) = e(B_u) - e(G) + \sum_{v \in A_u} d_G (v)$ (juntando ambas ecuaciones). Ahora, por la desigualdad $e(B_u) + e (A_u)\geq t$, se tiene que
\[
    e(B_u) \geq t - e(A_u) = t +e(G) - e(B_u) - \sum_{v \in A_u} d_G (v)
\]
y por lo tanto
\[
    2 e(B_u) \geq t + e(G) - \sum_{v \in A_u} d_G (v).
\]
Sumando sobre todos los $u \in V(G)$ y utilizando que $k_3 (G) = \frac 1 3 \sum_{u \in V(G)} e (B_u)$, concluimos:
\[
    k_3 (G) \geq \frac 1 2 \cdot \frac 1 3 (n t + n e(G) - \sum_{u \in V(G) } \sum_{v \in A_u} d_G (v));
\]
sin embargo, afirmamos que vale la siguiente igualdad:
\[
    \sum_{u \in V(G)} \sum_{v \in A_u} d_G (v) = \sum_{x \in V(G)} d_G (x) ( n - d_G (x));
\]
ya que cada término de la sumatoria se acota por $\frac n 2  \cdot (n - \frac n 2) = \frac{n^2}{2}$. De aquí concluimos el resultado.

Veamos la afirmación: notar que para cada $x \in V(G)$, su cantidad de aristas $d_G (x)$ es contada exactamente $\abs {A_x} = n - d_G (x)$ veces del lado izquierdo de la sumatoria
\end{proof}

Como corolario, se prueban los siguientes dos teoremas:
\begin{theorem}[Estabilidad]
Sean $n,t \in \naturals$, y $G$ es $K_3$-libre con $n$ vértices. Si $e(G) \geq \frac{n^2}{4}- t$, entonces $G$ contiene un grafo bipartito con al menos $e(G)-t$ aristas.
\end{theorem}
\begin{proof}
Si $G$ no tuviera un grafo bipartito con al menos $e(G) - t$ aristas, entonces $G$ estaría $(t+1)$-lejos de ser bipartito. Por el Teorema \ref{th:teorema de furedi si G esta t lejos de ser bipartito entonces tiene triangulos} tiene al menos
\[
    \frac{n}{6} \left ( e(G) - \frac{n^2}{4} + (t + 1) \right ) \geq \frac n 6
\]
triángulos, i.e., al menos uno, lo cual es absurdo.
\end{proof}

\begin{theorem}[Supersaturación]
Sean $n,t \in \naturals$, y $G$ un grafo con $n$ vértices. Si $e(G) \geq \frac{n^2}{4} + t$, entonces $G$ contiene al menos $t \cdot n /3$ triángulos.
\end{theorem}
\begin{proof}
Notar que $G$ está $t$-lejos de ser bipartito, en efecto, un grafo bipartito de orden $m \leq n$ tiene a lo más $\frac{m^2}{4}\leq \frac{n^2}{4}$ aristas, pero $G$ tiene al menos $\frac{n^2}{4} + t \geq \frac{m^2}{4} + t$ aristas. Luego por el Teorema \ref{th:teorema de furedi si G esta t lejos de ser bipartito entonces tiene triangulos}, $G$ tiene
\[
    \frac{n}{6} \left ( e(G) - \frac{n^2}{4} + (t + 1) \right ) \geq \frac{n}{3} t
\]
triángulos.
\end{proof}


\bigskip

\begin{theorem}[Füredi, 2015 -- Estabilidad]
Sean $n,k \in \naturals$, $t \geq 0$ y $G$ un grafo $K_{k+1}$-libre en $n$-vértices. Si $e(G) \geq t_k (n)- t$, entonces $G$ contiene un subgrafo generador $k$-partito con al menos $e(G)-t$ aristas.
\end{theorem}
\begin{proof}
Haremos inducción en $k$. El caso $k = 1$ tenemos que $t_k (n) = 0$ y siempre se cumple. Entonces supongamos que $k \geq 2$. Tomemos $u \in V(G)$ con $d_G (u) = \Delta (G)$. Definamos $G' := G[B]$ con $B = N_G (u)$. Sea $A = V(G) \setminus B$. El grafo $G'$ es $K_k$-libre porque $G$ es $K_{k+1}$-libre, luego por el Teorema de Turán \ref{th:teorema de Turan todo grafo K_k+1 libre en n vertices tiene cantidad de aristas e < t_k (n)}, $e(G') \leq t_{k-1} (d)$ con $d := \abs B$ y entonces podemos definir $t' := t_{k-1}(d) - e(G') \geq 0$ y aplicar hipótesis inductiva al grafo $G'$. Así, $G'$ contiene un subgrafo $H'$ generador $(k-1)$-partito con al menos $e(G') - t' = 2 e(G') - t_{k-1} (d)$ aristas.

 Probemos que
\[
    H := \Big (V(H') \cup A, E(H') \cup E(A,B) \Big)
\]
tiene al menos $e(G) - t$ aristas, y así $H$ es un subgrafo $k$-partito generador de $G$ con al menos $e(G)- t$ aristas. En efecto, queremos probar que
\[
    e(H') + e(A, B) \geq e(G) - t;
\]
como $e(G) = e(A,B) + e(G') + e(A)$, la desigualdad de arriba es equivalente a
\[
    e(H') \geq e(G') + e(A) - t \quad \Leftrightarrow \quad e(H') - e(G') + t \geq e(A).
\]
Ya que $e(H') \geq e(G') - t'$, nos queda que la última desigualdad es cierta si $e(A) \leq t - t'$.

Sabemos que
\[
     2 e(A) + e(A,B) = \sum_{v \in A} d_G (v) \leq d \cdot (n - d),
\]
donde la desigualdad sale de que la sumatoria tiene $(n-d)$ términos y cada grado $d_G(v) \leq \Delta (G) = d_G(u) = \abs B = d$;
y reemplacemos $e(A,B) = e(G) - e(A) - e(G')$ y nos queda
\[
    e(A) + e(G) - e(G') \leq d \cdot (n-d).
\]
Ahora, notar que
\[
    t_k (n) \geq t_{k-1} (d) + d \cdot (n-d),
\]
pues el lado izquierdo es la cantidad de aristas de un grafo de Turán (la cual es máxima) y el lado derecho es la cantidad de aristas de un grafo $k$-partito en $n$-vértices: el obtenido a patir del grafo de turán $T_{k-1} (d)$ agregando $n-d$ vértices y conectándolos a las $k-1$ particiones de $T_{k-1} (d)$. Juntando todo,
\[
    e(A ) \leq d \cdot (n-d) - \overbrace{e(G)}^{\geq t_k (n) -t} + \overbrace{e(G')}^{=t_{k-1} (d) - t'} \leq d \cdot (n-d) - t_k (n) + t + t_{k-1} (d) - t' \leq t - t'
\]
como queríamos probar.
\end{proof}



\section{Teorema de Erdös-Stone}


\begin{notation}
Notaremos por $K_s (t)$ al grafo de Turán $T_{s} (t \cdot s)$.
\end{notation}

\begin{theorem}[Erdös-Stone, 1946]
Sea $H$ un grafo con $e(H) \geq 1$. Entonces
\[
    \ex n H \leq \left ( 1 - \frac{1}{\chi (H) - 1}  + o (1) \right) \cdot \frac{n^2}{2} \quad (n \to \infty) \\
\]
\end{theorem}

\begin{obs}
Sea $H$ un grafo con $e(H) \geq 1$. Entonces
\[
    t_{\chi (H) -1} (n) \leq \ex n H,
\]
pues todo grafo $G$ necesita se al menos $\chi (H)$ coloreable para tener a $H$ incrustado, por lo tanto $T_{\chi (H) - 1} (n)$ no puede contener a $H$.
\end{obs}

\begin{obs}
\[
t_{\chi (H) - 1 (n)} \sim \left ( 1 - \frac{1}{\chi (H)- 1}\right ) \frac{n^2}{2}.
\]

Con lo cual, la desigualdad de Erdös-Stone es asintóticamente justa.
\end{obs}

\begin{lemma}
Sea $c \in (0,1)$ y sea $\varepsilon > 0$. Si $G$ es un grafo con $n$ vértices, con $n$ lo suficientemente grande tal que
\[
    e(G) \geq c \frac{n^2}{2},
\]
entonces existe un subgrafo $G' \subset G$ con
\[
    v(G') \geq \varepsilon n \quad \text y \quad \delta (G') \geq (c- \varepsilon) \abs{G'}.
\]
\end{lemma}
\begin{proof}
Sea $G_n, G_{n-1}, G_{n-2}, \ldots, G_t$ la secuencia de subgrafos de $G$ obtenida de la siguiente manera: $G_n := G$ y el grafo $G_{n - (i+1)}$ se obtiene a partir de $G_{n-i}$ borrando un vértice $v \in V(G_{n-i})$ con $d_{G_{n-i}} (v) < (c - \varepsilon) \cdot \abs{G_{n-i}}$; además, $G_t$ es el último grafo de la secuencia. Notar que $\abs{G_{n-i}} = n - i$.

Afirmamos que $t \geq \varepsilon n$ para $n$ lo suficientemente grande. Para eso, calculamos la cantidad total de aristas borradas para la obtención de $G_t$:
\[
    \sum_{i = t + 1}^n d_{G_{n-i}} (v_i) < (c-\varepsilon) \sum_{i = t+ 1}^n i,
\]
y como $G_t$ tiene a lo más $\binom t 2$ aristas, tenemos que
\[
    e(G) \leq (c - \varepsilon) \sum_{i = t + 1}^n i + \binom t 2 .
\]
A su vez, $e(G) \geq c \frac{n^2}{2}$.

Por un lado, $\sum_{i = t + 1}^n i = \frac{(n-t) (n+ t -1 )}{2} = \frac{n^2 - t^2}{2} + \frac{n - t}{2}$, así
\[
    e(G) \leq (c-\varepsilon) \frac{n^2 - t^2}{2} + \frac{n-t}{2} + \frac{t^2}{2},
\]
luego con la cota inferior de $e(G)$:
\[
    c \frac{n^2}{2} \leq (c-\varepsilon) \frac{n^2 - t^2}{2} + \frac{n-t}{2} + \frac{t^2}{2}.
\]
Juntando las $n$ del lado izquierdo y las $t$ del lado derecho:
\[
    \frac{n^2}{2} \varepsilon - \frac{n}{2} \leq \frac{t^2}{2} (1 - (c- \varepsilon)) - \frac t 2;
\]
consideremos la función creciente $f : \reals_{\geq 0} \to \reals$ dada por $f(x) := \frac{x^2}{2} (1 - (c- \varepsilon)) - \frac x 2$ (tenemos que $1 - (c - \varepsilon) > \frac 1 2$) para $\varepsilon$ suficientemente chico. Luego, $f(x) \leq f(y)$ si y solo si $x \leq y$ para todo $x,y \in \reals_{\geq 0}$. Por lo tanto, si vemos que $f(\varepsilon n) \leq \frac {n^2} 2 \varepsilon - \frac n 2$, se seguirá que $f(\varepsilon n) \leq f(t)$ y por lo tanto $\varepsilon n \leq t$. En efecto,
\begin{align*}
\frac{\varepsilon^2 n^2}{2} (1 - (c- \varepsilon)) - \frac{\varepsilon n}{2}
&= \frac{\varepsilon^2 n^2}{2} - \frac{(c- \varepsilon)}{2} \varepsilon^2 n^2 - \frac{\varepsilon n}{2} \\
&< \varepsilon \frac{n^2}{2} - \frac n 2.
\end{align*}
\end{proof}

\begin{lemma}
Para todo $r, t \in \naturals$ y $\varepsilon > 0$, existe $n_0 \in \naturals$ tal que si $G$ es un grafo con $n \geq n_0$ vértices y
\[
    \delta (G) \geq \left ( 1 - \frac 1 r + \varepsilon \right ) n
\]
luego $K_{r+1} (t) \subset G$.
\end{lemma}
\begin{proof}
Procedemos por inducción en $r$. Para $r = 1$, tenemos que $K_2 (t) = K_{t,t}$ y sabemos que en este caso $\ex n {K_{t,t}} = o (n^2)$. Como $n$ es lo suficientemente grande, $K_{t,t} \subset G$.

Ahora, supongamos que $r \geq 2$. Primero, encontraremos por hipótesis inductiva, una copia de $K_{r} (q)$ con $q \geq t/\varepsilon$. $A := \bigcup_{i = 1}^r A_i$. En efecto,
\[
    \left ( 1 - \frac 1 r + \varepsilon \right ) n \geq \left ( 1 - \frac 1 {r-1} + \varepsilon \right) n.
\]

Luego, definimos $X \subset B := V(G) \setminus A$, el conjunto de todos los vértices que tienen al menos $k$ vecinos en cada $A_i$ con $i \in [q]$. Mostramos que $\abs X \to \infty$ cuando $n \to \infty$. Para esto, acotamos $e(A,B)$ por abajo:
\begin{align*}
    e(A,B) = \sum_{v \in A}  d_G (v) - 2 e(A) \\
        &\geq q r  \left (1 - \frac 1 r + \varepsilon \right ) - 2 \frac{(q r)^2}{2}.
\end{align*}
Y a cotamos por arriba:
\[
    e(A,B) \leq \abs X  q r + ( \abs B - \abs X) ( q (r-1) + t - 1).
\]
Juntando ambas desigualdades, tenemos:
\begin{align*}
- n (q (r-1) + t - 1) + q r \left (1 - \frac 1 r + \varepsilon \right ) n - (q r )^2 &\leq \abs X ( q r - q (r-1) - t + 1) \\
    \underbrace{(-t + 1 + \varepsilon q)}_{> 0} n - q r^2 \leq \abs X \underbrace{(q - t + 1)}_{> 0}.
\end{align*}
Por lo tanto, se sigue lo que queremos cuando $n \to \infty$.

Finalmente, demostramos que existen conjuntos
\[
    B_i \subset A_i \text{ con } \abs {B_i} = t \text{ y $t$ vértices $x \in X$ que satisfacen } N_G (x) \supset B_i.
\]
Sea $x \in X$, existen a lo más $\binom q t$ formas de elegir $B_i^x$ en $A_i$, donde $B_i^x$ satisface $\abs{B_i^x} = t$ y $N_G (x) \subset B_i^x$. Si $\abs X > \binom q t ^r \cdot (t-1)$, entonces por el principio del palomar tenemos lo que queremos.
\end{proof}

\begin{proof}[Demostración del Teorema]
Observemos que $H$ está contenido en el grafo $\chi (H)$-partito, completo y con partes de tamaño $\abs H$, es decir, $T_{\chi (H)} (\chi (H) \cdot \abs H)$. Con lo cual, basta probar el teorema para $H' := T_{\chi (H)} ( \chi (H) \cdot \abs H)$. De hecho, probaremos que para $r := \chi (H)- 1$, $t \in \naturals$, $\forall \varepsilon > 0$, existe $n_0 \in \naturals$ tal que:
\[
    \ex n K_{r +1} (t) \leq \left (1 - \frac 1 r + \varepsilon \right ) \frac{n^2}{2} \quad (n \geq n_0).
\]

Vamos a tomar $r = \chi (H) - 1$ y $t = \abs H$ y $\varepsilon > 0$ arbitrariamente pequeño. Sea $n$ lo suficientemente grande y $G$ con $n$ vértices tal que
\[
    e(G) \geq \left ( 1 - \frac 1 r + \varepsilon \right ) \frac{n^2}{2}.
\]
Aplicamos el primer lema con $c = 1 - \frac 1 r + 2 \varepsilon$. Así, obtenemos un subgrafo $G' \subset G$ con
\[
    \abs{G'} \geq \varepsilon \quad \text y \quad \delta (G') \geq \left ( 1 - \frac 1 r + \varepsilon \right ) \abs{G'}.
\]
Como $n$ es lo suficientemente grande, $\varepsilon n \geq n_0$ y por el segundo lema $G'$ contiene a $K_{r+1} (t)$, y por lo tanto el resultado se sigue.
\end{proof}


\begin{definition}
$G$ está \textbf{$t$-cerca} de ser $r$-partito si existe un subgrupo $r$-partito de $G$ con al menos $e(G) - t$ aristas.
\end{definition}





\begin{theorem}[Teorema de Estabilidad de Erdös-simonovits]
Para todo grafo $H$ con $e(H) \geq 1$, para todo $\varepsilon > 0$ existe $\delta > 0$ tal que: si $G$ es $H$-libre en $n$-vertices y
\[
    e(G) \geq \left ( 1 - \frac {1}{\chi (H) - 1} - \delta \right) \binom n 2.
\]
Entonces $G$ está $(\varepsilon n^2)$-cerca de ser $(\chi (H) - 1)$-partito.
\end{theorem}
\begin{proof}
Haremos la demostración con $H = K_{r+1}$ y para $H$ general lo haremos con el Lema de regularidad:
\end{proof}

\begin{quote}
Para todo $\varepsilon > 0$ existe $\delta > 0$ tal que: si $G$ es $K_{r+1}$-libre en $n$-vértices y
\[
    e(G) \geq \left ( 1 - \frac 1 r - \delta\right) \overbrace{\binom n 2}^{\sim \frac {n^2}2},
\]
entonces $G$ está $(\varepsilon n^2)$-cerca de ser $r$-partito.
\end{quote}

\begin{lemma}
Sea $r \in \naturals$ y $\delta > 0$ y $n$ suficientemente grande. Si $G$ es $K_{r+1}$-libre con $n$ vértices y
\[
    e(G) \geq \left ( 1 - \frac 1 r - \delta^2\right) \frac{n^2}2,
\]
entonces existe $G' \subset G$ con $v (G') \geq (1 - \delta)n $ y
\[
    \delta (G') \geq \left ( 1 - \frac 1 r - \delta \right) \abs { G'}.
\]
\end{lemma}
\begin{proof}

\end{proof}

\begin{lemma}
Para todo $r \in \naturals$, para todo $\varepsilon > 0$, existe $\delta > 0$ tal que si $G$ es $K_{r+1}$-libre con $n$ vértices y
\[
    \delta (G) \geq \left ( 1 - \frac 1 r - \delta \right) n,
\]
entonces existe una partición $V(G) = A_0 \coprod A_1 \coprod \cdots \coprod A_r$ tal que $\abs {A_0} \leq \varepsilon n$ y $A_i$ son conjuntos independientes para todo $i \in \{0, \ldots, r\}$.
\end{lemma}
\begin{proof}
Si tomamos $\delta >0$ lo suficientemente pequeño, entonces $G$ contiene una copia de $K_r$ por el Teorema de Turán \ref{th:teorema de Turan todo grafo K_k+1 libre en n vertices tiene cantidad de aristas e < t_k (n)} (esto ocurre si $e(G) \geq \left ( 1 - \frac 1 {r-1} \right)\frac{n^2}{2}$; tomar $\delta < \frac 1 {r-1} - \frac 1 r$).

Sea $A$ un conjunto de vértices que induce un $K_r$ en $G$. Sean $B := V(G) \setminus A$ y $X := \set{v \in V(G) | \abs{N_G (v) \cap A} \leq r -2}$, vamos a mostrar que $X$ es pequeño.
\[
    \left ( 1 - \frac 1 r - \delta \right) n r - r (r-1) \leq e(A, B) \leq (r-1) (n-r) - \abs X \quad (\sum_{v \in A} d_G (v) = e(A,B) + 2 \abs A),
\]
manipulando la desigualdad, obtenemos:
\[
    \abs X \leq \delta n r.
\]
Tomando $\delta < \min \{\frac{\varepsilon}{r}, \frac 1{r-1} - \frac 1 r\}$, los consjuntos independientes son:
\[
    A_u = \{u\} \cup \{v \in B | vu \not \in E(G)\}
\]
para cada $u \in A$.
\end{proof}


\begin{exercise}
Utilizando los lemas 1 y 2, probar el teorema de estabilidad de Erdös-Simonits para $H = K_{r+1}$.
\end{exercise}

Sea $\varepsilon > 0$, tomar $\delta = (\delta ')^2$ donde $\delta'$ se obtiene del Lema 2 con $\varepsilon ' > \frac \varepsilon 2$. Por hipótesis
\[
    e(G) \geq \left ( 1 - \frac 1 r - (\delta ')^2 \right ) \frac{n^2}{2},
\]
entonces por el Lema 1: existe $G' \subset G$ con $n' := v(G' ) \geq ( 1 - \delta ')n$ y $\delta (G') \geq \left ( 1 - \frac 1 r - \delta' \right) v(G') = n'$. Por el Lema 2: para $\varepsilon ' < \frac \varepsilon 2$ se tiene que $A_0, A_1, \ldots, A_r$ partición con $\abs {A_0} < \varepsilon ' n ' \leq \varepsilon ' n$ y $A_i$ conjunto independiente para todo $i = 0 , \ldots, r$. Así, se pierden en total a lo más
\[
    \delta ' n ^2 + \varepsilon ' n^2 < \varepsilon n^2
\]
aristas.






\section{Ejercicios}









\section{Regularidad}

\begin{definition}
Una partición de un grafo $G$, $X, Y \subset V(G)$, entonces definimos la \textbf{densidad} de $(X,Y)$ como la cantidad
\[
    d(X,Y) = \frac{e(X,Y)}{\abs X \abs Y}.
\]
\end{definition}

\begin{definition}
Dado $\varepsilon > 0$. Sea una partición $A,B \subset V(G)$ de los vértices de un grafo $G$. Diremos que el par $(A,B)$ es \textbf{$\varepsilon$-regular} si para todo $X  \subset A$, $Y \subset B$ con
\[
    \abs X \geq \varepsilon A \quad \text e \quad \abs{Y} \geq \varepsilon \abs B
\]
tenemos
\[
    \abs{d (X,Y) - d(A,B)} \leq \varepsilon.
\]
\end{definition}

\begin{definition}
Sea $G$ un grafo. Una partición $V(G ) = V_0 \coprod V_1 \coprod \cdots \coprod V_k$ es una \textbf{equipartición}, si
\[
    \abs {V_0} \leq \abs{V_1} = \abs{V_2} = \cdots = \abs{V_k}.
\]
Al conjunto $V_0$ lo llamamos \textbf{conjunto excepcional}.
\end{definition}

\begin{definition}
Sea $G$ un grafo con $n$ vértices y $\varepsilon > 0$. Diremos que una partición $V(G) = V_0 \coprod V_1 \coprod \cdots \coprod V_k$ es \textbf{$\varepsilon$-regular}, si $\abs{V_0} \leq \varepsilon n$ y a lo más $\varepsilon k^2$ pares $(V_i,V_j)$ con $1 \leq i , j \leq k$ no son $\varepsilon$-regulares.
\end{definition}

\begin{theorem}[Lema de Regularidad de Szemerédi]
Para todo $\varepsilon > 0$, $m \in \naturals$, existe $M = M (\varepsilon, m)$ tal que para cualquier grafo, existe una partición $\varepsilon$-regular
\[
    V(G) = V_0 \coprod V_1 \coprod \cdots \coprod V_k
\]
con $m \leq k \leq M$.
\end{theorem}


















%%%%%%%%%%%%%%%%%%%%%%%%%%%%%%%%%%%%%%%%%%%%%%%%%%%%%%%%%%%%%%%%%%%

%import{nombre de carpeta/}{Nombre del archivo}
%\subfile{Apendice/Apendice.tex}





%--------------------------------
\newpage

\bibliographystyle{alpha}
\bibliography{main.bib}{}
%--------------------------------







\end{document}

